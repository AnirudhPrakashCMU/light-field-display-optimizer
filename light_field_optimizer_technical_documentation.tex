\documentclass[11pt]{article}
\usepackage[margin=1in]{geometry}
\usepackage{amsmath}
\usepackage{amsfonts}
\usepackage{graphicx}
\usepackage{hyperref}
\usepackage{listings}
\usepackage{xcolor}

\title{Light Field Display Optimizer: Technical Documentation}
\author{Enhanced Multi-Ray Sampling Implementation}
\date{\today}

\begin{document}

\maketitle

\section{Overview}

The Light Field Display Optimizer is a PyTorch-based system that optimizes display patterns for light field displays using multi-ray sub-aperture sampling and realistic optical physics simulation. The system optimizes display images to recreate target scenes as viewed through a complete optical system consisting of eye optics, tunable focus lens, and microlens array.

\section{Optical System Model}

\subsection{Eye Optics}
The human eye is modeled with the following parameters:
\begin{itemize}
    \item \textbf{Pupil diameter}: 4.0 mm
    \item \textbf{Retina distance}: 24.0 mm (from eye lens)
    \item \textbf{Retina size}: 10.0 mm effective imaging area
    \item \textbf{Focal length range}: 17.0 - 60.0 mm (accommodation)
\end{itemize}

\subsection{Multi-Ray Sub-Aperture Sampling}
The system implements realistic depth-of-field through multi-ray sampling:
\begin{align}
\text{Ray origins} &= \text{Pupil samples} \times N_{\text{rays}} \\
\text{Ray directions} &= \frac{\text{Pupil point} - \text{Retina point}}{|\text{Pupil point} - \text{Retina point}|}
\end{align}

Where $N_{\text{rays}} = 16$ rays per pixel for realistic blur simulation.

\subsection{Tunable Focus Lens}
A tunable lens is positioned 50.0 mm from the eye with:
\begin{itemize}
    \item \textbf{Diameter}: 15.0 mm
    \item \textbf{Focal range}: 10.0 - 100.0 mm
    \item \textbf{Thin lens equation}: $\frac{1}{f} = \frac{1}{s_o} + \frac{1}{s_i}$
\end{itemize}

\subsection{Microlens Array}
The microlens array parameters:
\begin{itemize}
    \item \textbf{Distance from eye}: 80.0 mm
    \item \textbf{Array size}: 20.0 mm × 20.0 mm
    \item \textbf{Pitch}: 0.4 mm (center-to-center spacing)
    \item \textbf{Focal length}: 1.0 mm
    \item \textbf{Total microlenses}: 2,500 circular lenses
\end{itemize}

\section{Display System}

\subsection{Light Field Display}
The optimizable display consists of:
\begin{align}
\mathbf{D} &= \{D_1, D_2, \ldots, D_{N_f}\} \\
D_i &\in \mathbb{R}^{3 \times H \times W}
\end{align}

Where:
\begin{itemize}
    \item $N_f = 10$ focal planes
    \item $H = W = 1536$ pixels (display resolution)
    \item Each $D_i$ represents RGB display image for focal plane $i$
\end{itemize}

\subsection{Focal Length Assignment}
Fixed focal lengths are assigned linearly:
\begin{align}
f_i = f_{\min} + \frac{i-1}{N_f-1}(f_{\max} - f_{\min})
\end{align}
Where $f_{\min} = 10$ mm and $f_{\max} = 100$ mm.

\section{Scene Definitions}

\subsection{Geometric Scenes}
Six geometric scenes with parametric object definitions:
\begin{align}
\text{Object}_j &= \{pos_j, size_j, color_j, shape_j\} \\
pos_j &\in \mathbb{R}^3 \text{ (world coordinates in mm)} \\
color_j &\in [0,1]^3 \text{ (RGB values)}
\end{align}

\subsection{Spherical Checkerboard Scene}
A MATLAB-compatible spherical checkerboard with:
\begin{align}
\text{Sphere center} &= [0, 0, 200] \text{ mm} \\
\text{Radius} &= 50 \text{ mm} \\
\text{Pattern} &= \text{1000×1000 grid, 50 pixels per square}
\end{align}

The checkerboard pattern is mapped using spherical coordinates:
\begin{align}
\phi &= \arctan2(Z, X) \\
\theta &= \arctan2(Y, \sqrt{X^2 + Z^2}) \\
\text{Square}_{i,j} &= \lfloor\frac{\theta_{\text{norm}} \times 999}{50}\rfloor + \lfloor\frac{\phi_{\text{norm}} \times 999}{50}\rfloor \bmod 2
\end{align}

\section{Ray Tracing Algorithm}

\subsection{Forward Ray Tracing}
The system uses forward ray tracing from retina through the complete optical system:

\begin{enumerate}
    \item \textbf{Retina sampling}: Generate $N_{\text{pixels}}$ retina points
    \item \textbf{Pupil sampling}: Generate $N_{\text{rays}}$ pupil samples per retina point
    \item \textbf{Eye lens refraction}: Apply thin lens equation
    \item \textbf{Tunable lens refraction}: Secondary lens refraction
    \item \textbf{Microlens selection}: Find nearest microlens for each ray
    \item \textbf{Microlens refraction}: Apply microlens optical power
    \item \textbf{Display sampling}: Sample from appropriate display image
\end{enumerate}

\subsection{Ray-Sphere Intersection}
For spherical checkerboard scenes, ray-sphere intersection:
\begin{align}
\mathbf{oc} &= \mathbf{ray\_origin} - \mathbf{sphere\_center} \\
a &= \mathbf{ray\_dir} \cdot \mathbf{ray\_dir} \\
b &= 2(\mathbf{oc} \cdot \mathbf{ray\_dir}) \\
c &= \mathbf{oc} \cdot \mathbf{oc} - r^2 \\
\text{discriminant} &= b^2 - 4ac \\
t &= \frac{-b + \sqrt{\text{discriminant}}}{2a}
\end{align}

\section{Optimization Process}

\subsection{Loss Function}
The optimization minimizes mean squared error between target and simulated images:
\begin{align}
\mathcal{L} = \frac{1}{HW} \sum_{i=1}^{H} \sum_{j=1}^{W} ||\mathbf{I}_{\text{target}}(i,j) - \mathbf{I}_{\text{simulated}}(i,j)||^2
\end{align}

\subsection{Optimization Algorithm}
\begin{itemize}
    \item \textbf{Optimizer}: AdamW with learning rate 0.02
    \item \textbf{Weight decay}: $10^{-4}$
    \item \textbf{Gradient clipping}: Maximum norm 1.0
    \item \textbf{Mixed precision}: Automatic mixed precision (AMP) for A100 acceleration
    \item \textbf{Iterations}: 100 per scene
\end{itemize}

\subsection{Multi-Scene Training}
The system optimizes seven distinct scenes:
\begin{enumerate}
    \item \textbf{Basic}: 3 colored spheres at different depths
    \item \textbf{Complex}: 4 multi-colored spheres in complex arrangement
    \item \textbf{Stick Figure}: 6 spheres arranged as humanoid figure
    \item \textbf{Layered}: 3 spheres at different depth layers
    \item \textbf{Office}: 4 spheres representing office objects
    \item \textbf{Nature}: 4 spheres representing outdoor scene
    \item \textbf{Spherical Checkerboard}: MATLAB-compatible spherical pattern
\end{enumerate}

\section{Output Generation}

\subsection{Training Progress Visualization}
For each scene, the system generates:
\begin{itemize}
    \item \textbf{Progress GIF}: 100 frames showing optimization evolution
    \item \textbf{Target image}: Ground truth scene appearance
    \item \textbf{Simulated image}: Optimized display output
    \item \textbf{Loss curve}: Convergence visualization
\end{itemize}

\subsection{Display Analysis}
\begin{itemize}
    \item \textbf{Display images}: What each focal plane shows (10 planes per scene)
    \item \textbf{Eye views}: What the eye sees for each display focal length
    \item \textbf{Focal length mapping}: 10-100 mm range across display planes
\end{itemize}

\subsection{Global Analysis}
\begin{itemize}
    \item \textbf{Focal length sweep}: 100 frames showing accommodation effects
    \item \textbf{Eye movement sweep}: 60 frames showing parallax effects
    \item \textbf{Focus calculation}: $d_{\text{focus}} = \frac{f \times d_{\text{retina}}}{f - d_{\text{retina}}}$
\end{itemize}

\section{Technical Implementation}

\subsection{Memory Optimization}
\begin{itemize}
    \item \textbf{Batch processing}: 4096 pixels per batch
    \item \textbf{GPU memory management}: Automatic cache clearing
    \item \textbf{Mixed precision}: FP16/FP32 automatic casting
    \item \textbf{Gradient accumulation}: Per-scene optimization
\end{itemize}

\subsection{Computational Complexity}
\begin{align}
\text{Rays per optimization} &= N_{\text{pixels}} \times N_{\text{rays}} \times N_{\text{iterations}} \\
&= 512^2 \times 16 \times 100 = 419,430,400 \text{ rays}
\end{align}

\subsection{Hardware Requirements}
\begin{itemize}
    \item \textbf{GPU}: NVIDIA A100-SXM4-80GB
    \item \textbf{Memory usage}: ~2.12 GB peak
    \item \textbf{Compute capability}: CUDA 11.8+
    \item \textbf{Storage}: 15 GB container disk
\end{itemize}

\section{Results and Validation}

The optimization system successfully generates:
\begin{itemize}
    \item \textbf{28 individual files}: 4 files per scene × 7 scenes
    \item \textbf{Progress tracking}: Every iteration recorded
    \item \textbf{Multi-ray sampling}: Realistic depth-of-field effects
    \item \textbf{Optical accuracy}: Physical ray tracing through complete system
\end{itemize}

\end{document}